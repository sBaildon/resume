\documentclass[a4paper,11pt]{article}

% Sans serif
\usepackage{caladea}
\usepackage[T1]{fontenc}

% Title spacing
\usepackage{titlesec}
\titlespacing*{\section}{0pt}{3ex}{1ex}
\titlespacing*{\subsection}{0pt}{4ex}{0pt}
\titlespacing*{\subsubsection}{0pt}{0pt}{0.5em}
\titleformat{\subsection}{}{\thesubsection}{}{\bfseries\normalsize}
\titleformat{\subsubsection}{}{\thesubsubsection}{}{\itshape\small}

\usepackage{hyperref}
\urlstyle{same}

\setlength{\parindent}{0pt}

\usepackage[english]{babel}
\usepackage{blindtext}

\usepackage[utf8]{inputenc}
\usepackage[top=2.5cm,bottom=3cm,right=3cm,left=3cm]{geometry}
%\usepackage[margin=3cm]{geometry}

\usepackage{multicol}
\setlength{\columnsep}{1cm}

% For list styling
\usepackage{enumitem}
\setitemize{itemsep=0em}

\pagenumbering{gobble}

% Line height
\renewcommand{\baselinestretch}{1.1}

\begin{document}

\large
\noindent\textbf{Sean Baildon}
\hfill
\noindent\textbf{Resume, \the\year}
\normalsize

\vspace{2ex}
\hrulefill{}
\vspace{1ex}

% Use asterisk for unbalanced columns
\begin{multicols*}{2}

\section*{Experience}
\subsection*{Software Engineer, BAE Systems AI}
\subsubsection*{June 2015 - present}
\begin{itemize}[leftmargin=*]
	\item Responsible for end-to-end software life cycles; design, implementation and maintenance
	\item Focusing on Java development, supplemented with Python, Ruby, JavaScript, and C++
	\item Reducing developer overhead by 90\% through automated provisioning
	\item Working on security-critical distributed systems expected to handle 10k+ concurrent users
	\item Developing web-based user applications in TypeScript, React
	\item Creating and maintaining AWS deployments
	\item Championing effective, long-term continuous integration and test frameworks
\end{itemize}

\subsection*{Research Associate, Lancaster Univ.}
\subsubsection*{May 2014 - March 2015}
\begin{itemize}[leftmargin=*]
	\item Evaluated performance and usability of \href{https://github.com/broadbent/opencache}{OpenCache}, an SDN HTTP caching platform
	\item Developed a network load balancer in Python leveraging OpenCache's API
	\item Contributed to research evaluating OpenCache's performance as a failover load balancer during HTTP adaptive streaming (\href{https://en.wikipedia.org/wiki/Dynamic_Adaptive_Streaming_over_HTTP}{MPEG-DASH})
\end{itemize}

\section*{Education}
\subsection*{Lancaster University}
\subsubsection*{October 2011 - June 2014}
Computer Science, BSc - 1st class honours\par

\section*{Skills}
Comfortable with Java, Python, JavaScript, Shell, and Lua. Familiar with Go, Ruby, C++, and Rust.\medskip

Strong Linux/macOS background. Versed with a variety of DevOps and automated provisioning tools including Docker, Kubernetes, Ansible, and Packer. Fluent in git.\medskip

Experience with web development. An engineer with an affinity for design and user experience.

\section*{Points of Interest}
\subsection*{sInterface}
\subsubsection*{https://ui.baildon.co}
A user interface for World of Warcraft
\subsection*{OpenCache: A software-defined content caching platform}
\subsubsection*{http://ieeexplore.ieee.org/document/7116129/}
A paper discussing novel contributions to running a CDN on commodity hardware

\section*{Contact and Profiles}
\noindent \begin{tabular}{@{}ll}
email & \href{mailto:hello@baildon.co}{hello@baildon.co} \\
web & \href{https://baildon.co}{baildon.co} \\
github & \href{https://github.com/sbaildon}{github.com/sbaildon} \\
twitter & \href{https://twitter.com/sbaildon}{twitter.com/sbaildon}
\end{tabular}

\end{multicols*}

\end{document}
